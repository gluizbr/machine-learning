\chapter{Introdução}\label{cap_intro}

Este trabalho consiste em aplicar o conhecimento de clustering adquirido na disciplina Tópicos: Aprendizado de Máquina, tendo assim como objetivo:

\begin{itemize}
	\item Escolha dois datasets rotulados.
	
	\item Realize a análise estatística, visualização e pré-processamento dos dados.
	
	\item Realize os experimentos criando duas bases de teste distintas:
	\subitem - considerando todos os atributos do dataset ;
	\subitem - selecionando alguns atributos e descartando outros;
	
	\item Aplique três métodos de clustering distintos nas duas bases acima.
	
	\item Para cada dataset , em cada uma das bases, analise os resultados
	segundo medidas de qualidade de clustering , usando índices de
	validação interna (SSW, SSB, silhueta, Calinski-Harabasz, Dunn e
	Davis-Bouldin) e externa (pureza, entropia, acurácia, F-measure ,
	ARI, NMI).
	
	\item Proponha uma maneira adicional de comparar os resultados obtidos
	além das medidas acima.
	
	\item Compare e interprete os resultados dos dois experimentos em cada dataset
\end{itemize}
