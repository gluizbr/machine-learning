\chapter{Introdução}\label{cap_intro}

Este trabalho consiste em aplicar o conhecimento de clustering adquirido na disciplina Tópicos: Aprendizado de Máquina, tendo assim como objetivo:

\begin{itemize}
	\item Escolha dois conjuntos de dados para trabalhar o problema de regressão. Separe cada dataset em conjunto de treinamento e conjunto de teste. Explique o seu critério de separação e o método utilizado.
	
	\item Você deverá implementar soluções para cada dataset usando:
	
	\subitem regressão linear (ou regressão múltipla)
	
	\subitem regressão polinomial
	
	\subitem SVR (use os kernels linear, sigmoide, RBF e polinomial)
	
	\subitem rede neural (MLP ou RBF).
	
	\item Descreva os parâmetros/arquiteturas de cada modelo.
	
	\item Compare os resultados (para treinamento e teste) com as medidas de desempenho SEQ, EQM, REQM, EAM e r2 , e verifique qual a melhor opção dentre os métodos implementados que melhor se ajusta a seus dados.
	
	\item Você deverá fazer a visualização dos dados originais com os dados ajustados em cada experimento, tanto para o conjunto de treinamento quanto para o de teste. Os gráficos devem conter títulos nos eixos e legenda. Comente os resultados encontrados na visualização.
\end{itemize}
