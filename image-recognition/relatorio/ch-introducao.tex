\chapter{Proposta Trabalho Final}\label{cap_intro}

Como proposta de trabalho final para a disciplina de aprendizado de máquina é proposto desenvolver um código utilizando da linguaguem python e alguns frameworks como pandas, para realizar o processamento de imagens de flores com o intuito de classifica-las de acordo com seu tipo i.e. processar uma imagem de uma flor e dizer se é um dente de leão ou uma margaria, etc.

\section{Dataset}

Para realizar o trabalho será utilizado um dataset encontrado na basse de dados Kaggle (https://www.kaggle.com/), onde o dataset se encontra em 
(https://www.kaggle.com/alxmamaev/flowers-recognition)
 este dataset possui cinco classes de tipos de flores, que estão separados em pastas, que são:

\begin{itemize}
	\item Margaridas
	
	\item Dente de leão
	
	\item Rosas
	
	\item Girassol

	\item Tulipa.	
\end{itemize}

\section{Algoritmos}

Para este trabalho é proposto ser utilizado dois tipos de algoritmos para realizar o treinamento e classificação das imagens, sendo eles, Convolutional Neural Networks (CNN) e Random Forest e comparar os resultados obtidos para se dizer qual é o melhor algoritmo para a tarefa em questão.

\section{Convolutional Neural Networks (CNN)}

Para se realizar o processamento das imagens com o algoritmo de CNN será utilizado a biblioteca Keras, que utiliza como core o TensorFlow, ela possibilita criar e treinar modelos de aprendizado profundo, sendo fácil de usar e fácil de estender.

\section{Random Forest}

Para se realizar o processamento das imagens com o algoritmo de Random Forest será utilizado a biblioteca Skylearn que possui uma classe implementada com o algoritmo em questão.

\section{Pré Processamento}

Para o pré-processamento de ambos os algoritmos será utilizado uma biblioteca do python cv2, que permite ler as imagens, redimensionar e descrever as imagens de uma maneira que possa ser utilizada em ambos algoritmos. E também para processar os arquivos nas pastas as biblotecas glob e os

\section{Validação}

Para validar qual dos dois modelos é o melhor para o problema apresentado serão utilizadas as métricas de precisão, recall, f1-score e suporte. Onde estas metricas serão executadas para ambos os algoritmos e comparado os resultados.