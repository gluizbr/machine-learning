\chapter{Desenvolvimento}\label{cap_desenv}

Para o desenvolvimento das atividades inicialmente foi escolhido uma base de dados. A base a ser utilizada corresponde a dados de \textit{pacientes} que estão confirmados com o vírus covid-19 (corona víŕus).

\section{Pré-processamento e Visualização}
Na base há dados categóricos e temporais, na base em questão alguns atributos numéricos foram removidos e outros foram discretizados, também foram gerados valores float para as datas que representam cada dia em questão.

\section{Atributos}

Os atributos utilizados do dataset foram idade, sexo, data de sintomas, data de entrada no hospital, data de confirmação do virus.

O primeiro passo no pre-processamento dos dados é remover registros vazios que não possuiam alguma informação e irá atrapalhar o resultado obtido utilizando a biblioteca pandas.

\section{Idade}
Após a remoção dos valores em branco o atributo idade, é realizado uma validação de algumas idades que são reais ou seja entre 1 e 99 ou então informações em branco (NaN) são mantidas, outros valores como "4000" é removido devido a não representar nenhuma idade válida.

Depois desse processo, os valores não numericos são preenchidos com a moda da idade.

Os dados foram discretizados utilizando 3 possíveis valores, dentre eles 0 para crianças com menos de 15 anos, 1 para adultos entre 15 e 64 anos e 2 para idosos com mais de 64 anos.

\section{Sexo}
Para o sexo foi é uma normalização dos dados onde possuiam 5 possíveis valores, Female, female, Male, male, NaN, os valores que começam com a primeira letra em maiusculo devem ser alterados para ser todos minusculos.

Os valores que estão vazios devem ser preenchidos com a moda dos valores para melhor visualização dos dados.

Após este processamento deve ser discretizado os dados com 0 para male e 1 para female

\section{Datas}
As três possíveis datas do dataset selecionado serão processadas da mesma maneira, as datas estão em sua maioria no formato dd.mm.YYYY, algumas estão com o dia e o mes trocado, outras estão com informações por extenso como "final de dezembro" sem uma data especifica.

As datas que possuem informações por extensos foram removidas do dataset para nao haver informações inconsistentes dentre os dados, as datas que possuiam o mes e o dia trocadas, foram alteradas para não serem descartadas de maneira desnecessária, visto que é possível indentificar essas datas.

Após a normalização das datas, foi gerado um valor numerico que representa cada data, e os valores faltantes foram preenchidos com a média destes valores númericos representativos

\section{Visualização dos Dados}
É gerado sobre os dados a descrição deles com todos os valores utilizados para se gerar um blox-pot.

Para visualizar os dados utilizando da biblioteca seaborn, e matplotlib, foram confrontados todos os dados e valores, gerando diversos graficos, alguns podem ser uteis na visualização das informações como entrada do hospital confrontando o inicio dos sintomas ou com a data de confirmacao confrontando a entrada no hospital é possível ver a linearidade dos dados, e de que a maioria das pessoas sentem os sintomas e depois são confirmados com a doença.

Também é gerado um gráfico utilizando também da biblioteca seaborn para visualizar por exemplo que a maior quantidade de pessoas infectadas com o corona são adultos, e alguns idosos, com a minoria sendo crianças. Também é possível visualizar que a maioria das pessoas indentificadas são homens.